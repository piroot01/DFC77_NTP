\documentclass{assignment}
\usepackage[TS1]{fontenc}
\usepackage[utf8]{inputenc}
\usepackage{erewhon}
% \inEnglish % Uncomment for assignment in English

\begin{document}

\PrintForm
    {Tomáš Macháček} % Jméno a příjmení studenta
    {4.B} % Třída
    {Emil Miler} % Vedoucí práce
    {Zdroj přesného času řízený radiovým signálem DCF77 a poskytování času pomocí ethernetu - NTP server} % Název 
    { % Pokyny k vypracování
        Cílem práce je sestrojení vlastního NTP serveru, který bude synchronizován radiovým signálem vysílače DCF77. Zvolením vhodného elektrického zapojení bude přijímán a demodulován. Následně na zvolené plaformě bude zprovozněn NTP server, jako zdroj přesného času NTP serveru bude použit demodulovaný signál DCF77. Synchronizovaný čas a status bude zobrazen na integrovaném displeji.
	}
	{}
    {https://github.com/MachacekTomas/DCF77_reciever_and_NTP_server} % URL repozitáře

\PrintSchedule
    {  % Říjen
        \item Výběr tématu maturitní práce (do~22.\,10.\,2021)
    }
    { % Listopad
        \item Zahájení stavby přijímače - návrh DPS, antény
    }
    { % Prosinec
        \item Stavba přijímače - DPS, anténa a kryt na přijímač
    }
    { % Leden
        \item Zahájení stavby druhé části - NTP server
    }
    { % Únor
        \item Finalizace NTP serveru a uvedení do provozu
    }
    { % Březen
        \item Odevzdání finální verze práce (do~25.\,3.\,2022)
    }

\end{document}
