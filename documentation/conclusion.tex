Tento projekt měl za cíl zjistit a dokázat, že za kvalitní NTP server, který je řízený rádiově, nemusíme
utrácet desítky tisíc korun. Komerční řešení téhož se prodává okolo 16000 Kč, viz Meinberg
NTP servers.

Další cíle, které se podařilo při jeho kompletaci splnit, je kompaktnost a jednoduchost. Zapojení zařízení
vyžaduje pouze napájení a propojení s anténou, toť vše. Také bylo dbáno, aby bylo pokud
možno všechno ochráněno, zejména proti elektrickým jevům. Zároveň bylo myšleno i na
pořizovací náklady zařízení, cena se s rezervou vejde do 1500 Kč.

Při realizaci se samozřejmě objevily i komplikace. Největší z nich, která zároveň
vyžadovala nejvíce úsilí, byl špatně navržený plošný spoj. To znamená, že oprava by
vyžadovala kompletní přepracování HATu. To způsobilo, že nefunguje ethernet konektor.
Minoritní záležitost, která je způsoben požadavkem vertikálního SMD USB C konektoru, který
není běžný a je velice obtížné jej sehnat. Náhradou se stal mini USB konektor, micro nebyl
zvolen kvůli konstrukční pevnosti. Bohůžel se při instalaci nenávratně poníčil OLED
displej, jeho náhrada nebyla realizována.

Nicméně je, dle mého názoru, projekt úspěšný.
