Závěrečná práce pojednává o NTP serveru, který je synchronizovaný s přijímaným časem. Čas
je poskytován německým vysílačem DCF77 v Mainflingenu.

První část se zabývá návrhem antény, která je schopna přijímat signál o kmitočtu 77.5kHz.

V druhé časti práce se rozebírá demodulaci signálu a převod na průmyslovou RS-485
sběrnici. Vše je součástí tzv. antěnní jednotky, obsahuje anténu, demodulační obvod a systém kontrolující kvalitu
signálu, což je jedna ze dvou hlavních komponent projektu.

Třetí část se týká hlavní jednotky, skládá se z jednodeskového počítače, Raspberry Zero 2,
na který je zhotoven HAT. Ten převádí demodulovaný signál z RS-485 na TTL sběrnici.

Poslední pasáž rozebírá softwarovou stránku hlavní jednotky, tudíž zprovoznění NTP
serveru a zobrazování aktuálního času na LCD displeji.

