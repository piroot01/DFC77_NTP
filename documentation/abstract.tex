Závěrečná práce pojednává o NTP serveru, který je synchronizovaný s přijímaným časem. Čas
je poskytován německým vysílačem DCF77 v Mainflingenu.

První část se zabývá návrhem antény, která je schopna přijímat signál o kmitočtu 77.5 kHz.

V druhé časti práce se rozebírá demodulaci signálu a převod na průmyslovou RS-485
sběrnici. Vše je součástí tzv. anténní jednotky, obsahuje anténu, demodulační obvod a systém kontrolující kvalitu
signálu, což je jedna ze dvou hlavních komponent projektu.

Třetí část se týká hlavní jednotky, ta je zodpovědná za zpracování DCF77 signálu a
hostování NTP serveru. Jádrem je Raspberry Zero 2, na které je navržený na zakázku
vyrobený HAT. Také rozebírá softwarevou stránku hlavní jednotky, instalací NTP. Též
komplikace vzniklé při sestavování.

Poslední pasáž se zaobírá nasazením zařízaní
do provozu a posléze připojením klientů na server.
