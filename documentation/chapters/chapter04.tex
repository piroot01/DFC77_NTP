V závěrečné kapitole si podrobně popíšeme, jak zažízení nasadit. V podstatě, co s
zařízením dělat, když se nám dostane do rukou.

\section{Nasazení zařízení na lokální síť}
\subsubsection{1. Připojení}
    Jako první musíme zažízení, resp. hlavní jednotku, připojit k lokální síti. Provedeme
    to headless způsobem, takže první vyjmeme SD kartu z RPi. K SD kartě se dostaneme tak,
    že povolíme dva krajní M3 šrouby na zadní straně hlavní jednotky. Vysuneme opatrně
    uvolněný obal. Nikoliv se nesnažíme oddělit dvě části úplně, postačí jenom odklopit
    odšroubovanou část a uvidíme RPi navrchu. Poté vytáhneme SD kartu, možná budou
    překážet šrouby - odstraňte je proto zcela.

    Vsuneme SD kartu do počítače a vytvoříme soubor (pokud již existuje, pak editujeme)
    \verb|/etc/wpa_supplicant/wpa_supplicant.conf| a do něj zapíšeme.

    \begin{lstlisting}

    network={
        ssid="SSID"
        psk="password"
        key_mgmt=WPA-PSK
    }

    \end{lstlisting}

    Poté vrátíme SD kartu do zařízení přiklopíme obal, zašroubujeme, dbáme, abychom
    neskříply žádné vodiče. Zapojíme zařízení na napájení. Abychom zjistili, jakou adresu
    má zařízení můžeme použít například \verb|nmap -n 192.168.0.0/24|. Zařízení bude mít
    pravděpodobně jako jedniné otevřený jenom ssh port. Poté se jednoduše připojíme přes
    ssh jako uživatel \textit{ntpcli} \verb|ssh ntpcli@192.168.X.X|. Heslo je ntp. A nebo se zkusíme
    připojit přes \verb|ssh ntpcli@raspberrypi.local|.

\newpage

\subsubsection{2. Status antény}

    Nyní se zaměříme na NTP server. Jako první ověříme, že je nastartovaný NTP daemon.

    \begin{lstlisting}

    $ sudo systemctl status ntp
    # password for sudo is ntp
    $ status

    \end{lstlisting}

    \noindent Poslední příkaz používáme k zjištění stavu, ten by měl iniciálně vypadat takto

    {\scriptsize

    \begin{lstlisting}

    ntpcli@raspberrypi:~ $ status
    associd=0 status=0000 no events, clk_unspec,
    device="RAW DCF77 CODE (TimeBrick)", timecode="", poll=3, noreply=0,
    badformat=0, baddata=0, fudgetime1=877.000, stratum=0, refid=DCFa,
    flags=0, refclock_time="<UNDEFINED>", refclock_status="",
    refclock_format="RAW DCF77 Timecode",
    refclock_states="*NOMINAL: 00:02:48 (100.00%); running time: 00:02:48"

    \end{lstlisting}
    }

    a když je přijímán korektní signál a je správně dekódován (obvykle do tří minut po
    zapnutí - záleží na kvalitě signálu), potom takto:

    {\scriptsize

    \begin{lstlisting}

    ntpcli@raspberrypi:~ $ status
    associd=0 status=0020 2 events, clk_unspec,
    device="RAW DCF77 CODE (TimeBrick)",
    timecode="--#--##---#-------M-S12--1--P12---2P--4--2--412----2---2--P",
    poll=8, noreply=0, badformat=2, baddata=0, fudgetime1=877.000, stratum=0,
    refid=DCFa, flags=0,
    refclock_time="e5e76b81.00000000  Thu, Mar 24 2022 22:13:21.000",
    refclock_status="TIME CODE; (LEAP INDICATION; CALLBIT)",
    refclock_format="RAW DCF77 Timecode",

    \end{lstlisting}
    }

    Pokud trvá načtení správného bufferu dlouho, tak je slabý signál, což status antény
    neobjeví. Buď lze počkat, NTP server nevyžaduje synchronizaci každou minutu, postačí
    bohatě jedna synchronizace za 12 hodin, a nebo vyhledat místo s lepším příjmem.

    Pokud chceme vidět aktuální dekódovaný čas, můžeme využít funkci \verb|t|, na konzoli,
    jedná se o předvytvořený alias. Ta spustí
    \verb|watch| se sekundovou aktualizací. Vidíme-li \verb|Error 666: Check NTP status.|,
    znamená to, že hodiny nepřijímají. Pakliže se objeví aktuální čas, podařilo se
    dekódovat načíst buffer pro probíhající minutu.

\subsubsection{3. Nastavení na lokální síť}

    Tato část se převážne věnuje konfiguračnímu souboru NTP, \verb|/etc/ntp.conf|.
    Znamenaje, že nastavíme jací uživatelé budou mít přístup k času NTP serveru, resp.
    subnet. To provedeme editací 53 řádku konfiguračního souboru.

    \begin{lstlisting}
    restrict 10.42.0.0 mask 255.255.0.0 notrap nomodify
    \end{lstlisting}

    Na tomto příkladu vidéme, že k času budou mít přístup uživatelé pouze subnetu
    10.42.0.0 (například uživatel 10.42.6.9 bude mít přístup). A na konec editujeme řádek
    58, který říká jakým uživatelúm má NTP server vysílat.

    \begin{lstlisting}
    broadcast 10.42.255.255
    \end{lstlisting}

    \noindent To je vše, co se týká konfigurace NTP serveru.

\section{Nastavení klienta - chrony}

    Tato část je jenom takovým dodatkem pro uživatele operačního systému Linux. Postačí do
    konfiguračního souboru chrony \verb|/etc/chrony.conf| přidat:

    \begin{lstlisting}
    server 10.42.0.1 iburst
    \end{lstlisting}

    \noindent Kde 10.42.0.1 je adresa NTP serveru \cite{chrony}.
