\section{Teorie}
\subsection{Network time protocol}

NTP je protokol pro synchronizaci vnitřních hodin počítačů po paketové síti s proměnným zpožděním. Tento protokol zajišťuje, aby všechny počítače v síti měly stejný a přesný čas. Byl obzvláště navržen tak, aby odolával následku proměnlivého zpoždění v doručování paketů.

Počítač, který chce synchronizovat své hodiny, pošle pár dotazů několika NTP serverům a ty mu v odpovědi pošlou svůj, přesný čas. Klient z odpovědí nejprve vyloučí servery se zřejmě nesmyslným časem (s odchylkou 1000 sekund a více). Poté ponechá skupinu serverů s největším společným průnikem.

NTP je jeden z nejstarších dosud používaných IP protokolů. NTP původně navrhl Dave Mills z univerzity v Delaware a stále jej, spolu se skupinou dobrovolníků, udržuje. Současná verze je NTP verze 4, kterou popisuje RFC 5905.

NTP démon je uživatelský proces, který na stroji běží trvale. Většina protokolu a inteligence je implementována v tomto procesu. Pro dosažení nejlepšího výkonu je důležité, aby jádro operačního systému umělo řídit čas fázovým závěsem, místo aby přesný čas do systémových hodin dosazoval NTP démon přímo. Všechny dnešní verze Linuxu fázový závěs implementují.

NTP používá hierarchický systém „strata hodin“, kde systémy se stratem 1 jsou
synchronizovány s přesnými externími hodinami jako třeba GPS nebo jiné hodiny řízené
rádiovým signálem (v Česku obvykle DCF77). NTP systémy strata 2 odvozují svůj čas od
jednoho nebo více systémů se stratem 1 atd. To zabraňuje vzniku cyklu v grafu
synchronizujících se strojů. Stratum systému leží v rozsahu 1 až 14; stratum 0 mají
samotné referenční hodiny připojené k nejpřesnějšímu serveru; stratum 15 má počítač, který
se v důsledku výpadku sítě nemůže synchronizovat s zdrojem času, nebo se synchronizuje po
výpadku spojení \cite{ntp}.

\section{Hardware}

    V této části se budeme zabývat, hardwarovou stránkou hlavní jednotky. Bylo nutné
    vyřešit dvě záležitosti, za prvé jakou platformu zvolit pro běh NTP serveru a za druhé
    jak konvertovat zpět RS-485 signál. Jelikož provoz NTP serveru neni hardwarově
    náročný, bylo zvoleno Raspberry ZERO 2. To přineslo řadu výhod, okamžitě se vyřešil
    druhý problém, všechny přídavné obvody budou součástí tzv. HATu. A navíc je celá
    sestava velice kompaktní.
\subsection{RPi HAT}
    HAT je označení pro PCB, které je namontováno na, v našem případě jednodeskový
    počítač. Přičemž rozšiřuje funkčnost desky, přidává vlastnosti. Raspberry disponují
    2x20 pinovou lištou, GPIO, která je vhodná pro komunikaci s HATem. Rozměry HATu jsou
    shodné s Raspberry, je to z toho důvodu, aby byl HAT pomocí sloupků spojen s
    raspberry, a tím bylo zajištěno pevné spojení desek.

    Hlavní funkce HATu je jasná, má za úkol převádět RS-485 komunikaci zpět na TTL
    logickou úroveň. Ale potom byl na desku ještě přidán ethernetový převodník. Iniciálně
    se totiž plánovalo použít raspberry zero, to se neuskutečnilo, více bude řečeno
    později.
\subsubsection{Komunikace s anténou}

    Je použito naprosto stejné zapojení jako na řidicí jednotce antény, to znamená i
    stejný integrovaný obvod. Akorát je zapojený v režimu receiver. Výstup je zaveden na
    UART na headeru raspberry.
\\

\textbf{Zprovoznění UART komunikace}
\\

    Při zprovozňování serveru vyvstal problém. Na samé piny (10 a 12) je připojena
    sběrnice bluetooth. A přitom raspberry má několik UART zařízení, konkrétně AMA0,
    USB0 a S0. Pro naše účely je vhodné pouze AMA0, jelikož jako jediný disponuje
    nezávislým UART převodníkem, mimo CPU. Ostatní jsou závislé na aktuální taktovací
    frekvenci CPU, která se mění, a to je nežádoucí u přesného vyčítání dat.

    Takže jako první se deaktivoval kompletně bluetooth, přes \verb|raspi-config|. Ve
    stejném prostředí byl povolen AMA0 serial port.

    \vspace{1em}

    \begin{lstlisting}
        /boot/config.txt

        [all]

        enable_uart=1
        dtoverlay=disable-bt
    \end{lstlisting}

    \vspace{1em}

   Poté byla ověřena komunikace pomocí \verb|# screen /dev/ttyAMA0 9600|. Na konzoli by
   se měli zobrazovat znaky (jedno jaké) s vteřinovou periodou.

\newpage

\subsection{Zprovoznění DCF77 jako stratum 1 NTP serveru}
    Potom, co jsme zajistili funkční vyčítání dat z anténý správným UART rozhraním, se
    můžeme přesunout na samotnou instalaci NTP serveru \cite{ntporg}.
    \\

    \textbf{Instalace s podporou RAW DCF}
    \\

    Prvním krokem je nainstaoování NTP, avšak pro naše účely je nutné podpora DCF77.
    Instalaci provedeme dle následujících kroků.

    \vspace{1em}

    \begin{lstlisting}
    $ wget http://www.eecis.udel.edu/~ntp/ntp_spool/
    ntp4/ntp-4.2/ntp-4.2.8p15.tar.gz
    $ tar zxf ntp-4.2.8p15.tar.gz
    $ cd ntp-4.2.8p15/
    $ ./configure --enable-RAWDCF --prefix=/usr
    $ make
    $ sudo systemctl stop ntp.service
    $ sudo make install:
    $ echo "ntp hold" | sudo dpkg --set-selections
    \end{lstlisting}

    \vspace{1em}

    Poslední příkaz slouží k zakázání aktualizací NTP. Verzi NTP ověříme přes
    \verb|ntpq --version|.
    \\

    \textbf{Nastavení strata 1}
    \\

    Ješte před tím, než nastavíme referenční hodiny pro NTP, přidáme uživatele \verb|user| do skupiny
    \verb|tty|.

    \vspace{1em}

    \begin{lstlisting}
    $ sudo adduser user tty
    $ cat /etc/group | grep tty
    \end{lstlisting}

    \vspace{1em}

    Nyní zeditujeme wrapper kód NTP, tak, aby server užíval DCF jako zdroj stratum 1.
    Referenční hodiny \verb|/dev/refclock-0| jsou systémovým linkem na
    \verb|/dev/ttyAMA0|.

    \vspace{1em}

    \begin{lstlisting}
    #for using the DCF77 receiver add this
    if [ ! -L /dev/refclock-0 ]; then
            ln -s /dev/ttyAMA0 /dev/refclock-0
    fi

    #commented out this
    #if [ -e /run/ntp.conf.dhcp ]; then
            #NTPD_OPTS="$NTPD_OPTS -c /run/ntp.conf.dhcp"
    #fi
    \end{lstlisting}

    \vspace{1em}

\newpage

    Nastartujeme daemon \verb|sudo systemctl start ntp.service|. Můžeme zkontolovat
    přítomnost symbolického linku.


    \begin{lstlisting}

    $ ls -ahl /dev/ref*
    /dev/refclock-0 -> /dev/ttyAMA0

    \end{lstlisting}

    \vspace{1em}

    Dále upravíme soubor \verb|/etc/ntp.conf|.

    \begin{lstlisting}

    ntp.conf

    \end{lstlisting}

    \vspace{1em}

    Konečně restartujeme NTP daemon a zkontrolujeme aktivitu NTP serveru, a to přes
    \verb|ntpq -p|. Správný příjem lze jednoduše ověřit tak, že hodnota \textit{reach} je
    377.
    \\

    \textbf{Oprava časové diletace = fudgetime}
    \\

    Díky nedokonalosti přenosu a časovým prodlevám vzniklé konverzí, je záhodné nastavit
    kompenzační časovou konstantu, která bude vyrovnávat časový schodek. Celkový fudgetime
    je dán součtem předkonfigurovaným a reálného, ten je v našem případě 592 ms.
    Předkonfigurovaný fudgetime zjistíme následovně (zde 292 ms).

   \begin{lstlisting}

    $ ntpq -c cv
    ...
    ... fudgetime1=292.000,
    ...
   \end{lstlisting}

\subsection{Ethernetové připojení a USB-C napájení}

Tyto dva hardwarové prvky se nepodařilo zprovoznit. Ethernet konektor slouží k navázání
stabilního připojení. Problém spočíval v tom, že se zapomnělo připojit jednu nožičku
integrovaného obvodu ethernetového převodníku na +3.3V, jednalo se o napájení TX. Bypass
se povedl, obvod fungoval bez problémů. Avšak kvůli veliké křehkosti a nestabilnosti spoje
byl čip odpájen a změněn návrh plošného spoje, který se nestihl znovu poptat.

S napájecím konektorem byl nenapravitelný problém. Vlastně hlavní význam integrovat
napájecí konektor na desku byla špatná orientace napájecího konektoru na RPi. A navíc
USB-C napájení je v dnešní době nejuniverzálnější konektor. Problém spočíval v špatném
footprintu. Vyřešilo se to tak, že se na napájecí linky RPi byly napájeny vodiče, které se
přivedly na USB mini konektor, je robustnější než micro.
