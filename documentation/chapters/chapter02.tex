\section{Teorie}
\subsection{Amplitudová modulace}
    Amplitudová modulace, zkráceně AM, je nejjednodušší a nejstarší typ modulace. V
    závislosti na změně modulačního signálu se mění amplituda nosného signálu.
    Nosná vlna má řádově vyšší frekvenci než modulovaný signál. Fáze a frekvence se u této
    modulace nemění.
\\

    \textbf{Matematický popis}
\\

    Pokud nosnou vlnu vyjádříme jako $n(t) = N \sin{(\Omega t)}$, kde $N$ je amplituda a
    $\Omega$ je úhlová frekvence, a jednoduchý harmonický signál $m(t) = M \sin{(\omega t
    + \phi)}$, kde $\phi$ je fázový posun vůči nosné $n(t)$, pak AM vznikne složením
    přidáním $m(t)$ k amplitudě nosné $N$. Dostáváme tedy

    \begin{equation}
        y(t) = (N+M \sin{(\omega t +\phi)})\sin{(\Omega t)}
    \end{equation}
\subsection{DCF77}
    Označení DCF77 se užívá pro německý dlouhovlnný rádiový vysílač a také jedinou
    stanici, jíž poskytuje. Nosná frekvence DCF77 je 77500 Hz. Antény stanice se nacházejí
    v Mainflingenu. Pravidelné a nepřetržité vysílání začalo v roce 1970 \cite{dcf77}.
\\

    \textbf{Vysílač}
\\

    Vysílač má výkon 50 kW, avšak skutečný vyzářený výkon je přibližně 25 kW. K vysílání
    je použitá všesměrová anténa s kapacitním nástavcem, hlavní anténa je vysoká 150 m,
    přičemž při údržbách se používá 200 m vysoká záložní. Dosah vysílače je 1500 až 2000
    km. V České republice fungoval do roku 1995 velice podobný vysílač OMA \cite{dcf}.
\\

    \textbf{Kódování DCF77}
\\

    Kódování časové informace je prováděno pulzně šířkovou modulací (základní AM), poklesem amplitudy
    nosné na 25 \% na začátku každé sekundy. Pakliže je puls dlouhý 100ms jedná se
    logickou 0, 200ms dlouhý je logická 1. Aktualizace přenášené informace je prováděna
    každou minutu.

    Přenášejí se tyto informace:

    \begin{itemize}
        \item čas platný pro následující minutu
        \item datum platné pro následující minutu
        \item číslo dne v týdnu
        \item hlášení změny časové zóny 1 hodinu předem (informace o přechodu ze standardního na letní čas nebo opačně)
        \item časová zóna (je-li aktivní CET nebo CEST)
        \item hlášení přestupné sekundy 1 hodinu předem
        \item provoz na normální a rezervní anténu
        \item zabezpečení přenosu – několik paritních bitů
        \item od roku 2007 i předpověď počasí
    \end{itemize}

\subsection{RS-485 komunikace}
    Jedná se o typ komunikace, která se především používá v průmyslových prostředích.
    Standard 485 je navržen tak, aby umožňoval vytvoření dvouvodičového poloduplexního
    vícebodového sériového spojení. Vlastně se jedná o značně napěťově posílenou RS-232, což je
    totéž co RS-485, avšak dosah sběrnice je až 1200 m, oproti 20 m.

    Komunikační linka je tvořena dvěma vodiči (označené A a B), na obou koncích jsou kvůli
    zvýšení odolnosti proti rušení 120$\Omega$ rezistory. Maximální přenosová rychlost je
    10 Mib/s.

    Přenos dat je uskutečněn pomocí 7 nebo 8bitových rámců se startbitem a poté jedním či
    více stopbity. Vysílač by měl na výstupu při logické 1 (klidový stav linky) generovat
    na vodiči A napětí −2 V, na vodiči B +2 V, při logické 0 by měl na vodiči A generovat
    +2 V, na vodiči B −2 V \cite{rs485}.

\section{Zpracování anténní jednotky}
Mezi hlavní požadavky anténní jednotky patřila i kompaktnost, proto byl vytvořený, na zakázku,
plošný spoj, na nemž jsou uložené všechny komponenty. Dělí se na tři základní sektory,
první je zodpovědný za demodulaci signálu, druhý převádí demodulovaný signál na RS-485
sběrnici a poslední vyhodnocuje jakost signálu. Každému sektoru je přiřazen jeden
integrovaný obvod.
\subsection{Demodulace DCF77 signálu}
    Pro tento účel byl zvolen U4224B integrovaný obvod. Jedná se o přijímací obvod, který
    disponuje demodulací od 40 do 80 kHz, podle zvoleného krystalu. Má velmi vysokou
    citlivost, takže i v oblatech s vysokým rušením či slabým signálem, je možně anténu
    dobře naladit.

    Mohlo být použito i zapojení typu superhat. Oscilátor naladěný na frekvenci 77 kHz by
    byl přiveden na vstup vyváženého směšovače, kde společně s DCF77 signálem (77.5 kHz),
    by vytvořil signál o rozdílovém kmitočtu 500 Hz. Tento signál by byl přiveden na
    vstup operačního zesilovače, který by fungoval jako aktivní pásmová propust s šířkou
    pásma 50 Hz. Jelikož by obvod nebyl stabilní, bylo by také nutné vybavit obvod AVC
    (aktivním vyvážením citlivosti).

    Obvod výše popsaný by bylo možné sestavit pomocí diskrétních součástek, ale byl by
    složitý. Situaci navíc komplikuje fakt, že dnes hodně spínaných zdrojů generují
    kmitočty, které vytvářejí harmonické násobky, a ty ruší příjem. Proto šířka pásma 50
    Hz je prostě moc široká. Zúžení pásma vedlo k použití tzv. krystalového filtru.
    Implementace krystalového filtru vedla na integrovaný obvod.

    Na výstupu integrovaného obvodu je tedy demodulovaný DCF77 signál. Kvůli zvýšení
    stability a ochrany dekodéru byl k obvodu zahrnut tvarovač signálu. Je to obvod, který
    se skládá z tranzistoru, který chrání dekodér a Schmittův klopný, obvod na definování
    napěťové úrovně. Tento signál je vedený na vstup převodníku RS-485.

\subsection{Převod na RS-485 sběrnici}
    Pro dobrý příjem signálu je nutné, aby byla anténa na vyvýšeném místě, bez rušení.
    To znamená, že napájecí vodiče, ale i datové linky, musejí být alespoň 10m dlouhé.
    Napájení není problém, ale sigál vedený klasickou TTL komunikací by byl ohrožen,
    norma TTL deklaruje dosah 2-3 metry. Tím pádem bylo nutné nalézt jinou normu.

    Potom byla zvolena norma RS-232, ta má dosah 20m. Ale potom se přešlo na RS-485, ta má
    jednak o 1 km vetší dosah a používá stejný typ kódováni jako 232. A to z důvodu jednak
    stejné požizovací ceny a vyšší napěťová úroveň zajišťuje značně vyšší odolnost vůči
    rušení.

    Pro transkódování signálu byl použit integrovaný obvod SN75176, nepatří mezi rychlé
    převodníky, avšak na naší aplikaci je postačující. Komunikace probíhá jednostranně
    halfduplexem.

    Abychom ochránili jak anténní jednotku, jsou za obvodem zapojeny trisily a teplem
    vratné pojistky proti přepětí a nadproudu, který se může indukovat ve vodičích.
    Nakonec je signál zaveden na vstup Ethernet konektoru. Ten byl použit proto, že
    má pojistku a oproti ostatním průmyslovým konektorům je velice levný. A navíc UTP
    kabel využívá kroucenou linku, která je záhodná u 485 komunikace.

\subsection{Systém pro vyhledávání signálu}

    Abychom nemuseli při ladění antény čekat dvě minutu na načtení celého bufferu a potom
    zjišťovali, zda je správný. Nutno podotknou, že anténa je směrová, takže záleží na
    směru normálového vektoru vůči zdroji, což je anténa DCF77. Což ještě ztěžuje situaci
    ladění. Další, rychlejší, způsob ladění vyžaduje osciloskop. Proto bylo nutné vybavit
    obvod zařízením, které by v reálném čase sledovalo přijímaný signál a vyhodnocovalo
    jeho kvalitu.
    \newpage
    Srdcem je mikrokontrolér PIC, na který je nahraný program, který analyzuje signál z
    přijímače. Dále jsou na PIC připojeny dvě LED, které fungují jako signalizace. Jedna
    je zeléná, značící, že je obvod napájen a druhá, oranžová, signalizuje kvalitu
    signálu. A to následujícím způsobem:
    \vspace{1em}

    \-\hspace{0cm} \textit{Pokud dioda nesvítí, potom je signál zcela vadný. Jakmile se dioda
    rozbliká, znamená to, že byl přijat validní bit, pakliže se dioda rozbliká rychleji (s
    vyšší frekvencí), byl přijat další validní bit. Tento postup pokračuje, až dioda
    svítí, neboli anténa je naladěná. S klesající frekvencí se naopak anténa rozlaďuje.
    Takže svítící dioda je zárukou dobrého signálu. }
    \vspace{1em}

    Program funguje tak, že kontroluje správnou délku signálu, frekvenci, intervali mezi
    jednotlivými pulsy, inverzní délku pulzů a minutový puls (prázdný). Pakliže všechny
    parametry jsou splněny je bit zaznamenán jako korektní. Tuto proceduru provádí s
    každým bitem a to v reálném čase, takže uživatel má aktuální přehled o kvalitě
    přijímaného signálu a může nalézt místo s nejlepším příjmem.

    Deska je osazená pinovou lištou, která slouží k flashování programu, takže je velice
    snadné aktualizovat program v PICu.
