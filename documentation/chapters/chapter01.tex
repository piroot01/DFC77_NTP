\section{Teorie}

Anténa je zařízení, které slouží k vysílání či přijímání elektromagnetického záření.
Každý vodič, jímž prochází střídavý elektrický proud, je vlastně anténa. Geometrickým
tvarováním vodiče a přidáváním jistých materiálů (ferity) se upravuje účinnost antény, dle
vhodných parametrů.

Každá anténa má své základní parametry, mezi hlavní patří zisk, rezonanční frekvence,
šířka přijímaného pásma a vyzařovací úhel. Podle použití antény se tyto parametry mění.
Avšak hlavním parametrem, který nejvíce ovlivňuje vlastnosti, je fyzický tvar antény.

Proto dělíme antény na dipólové, monopólové, složené (Yagi-Uda anténa), smyčkové, kónické
a clonové. Každá kategorie představuje jiné spektrum použití. Dipólové
se používají pro vysílání TV signálu. Monopólové se používají při
příjmu rádiových stanic, vlnové délky řádu desítek centietrů. Složené jsou vysoce účinné
pro širokopásmový příjem, TV i rádia. Smyčkovým anténám se budeme ještě věnovat. Kónické
jsou vysoce účiné při vysokých frekvencích. Clonové jsou ideální pro dlouhý dosah.

\subsection{Smyčková anténa}

Smyčkové antény jsou velice jednoduché a všestranné. Jejich tvar je rozmanitý, například
zaujímají tvar čtvercovitý, trojúhelníkový, kruhový a elipsovitý. Kvůli jednoduchému
návrhu a analýze jsou nejvíce rozšířené mezi radioamatéry.

Klasifikují se dvou hlavních skupin, elektricky malé a velké. Elektricky malé antény jsou
ty, jejichž celková délka (počet závitů znásobený obvodem antény) je menší než desetina
vlnové délky. Zatímco velké antény svojí délkou odpovídají vlnové délce rezonanční
frekvence. Nejvíce se smyčkové antény používají.

Elektricky malé smyčkové antény mají obvykle menší radiační odpor než ztrátový odpor, tedy
jsou špatné zářiče, používají se výhradně na příjem signálu. Nejvíce se používají při
radiovému příjmu.

Radiační odpor lze zvýšit, tím pádem zvýšit zisk antény, elektrickou délkou antény nebo
použít jádro z materiálu o vysoké permeabilitě, nejčastěji ferit. Anténa, která je tvořená
cívkou navinutou na feritovém jádře, se nazývá feritová smyčková anténa.

\newpage

\section{Konstrukce}



\subsection{Program pro výpočet antény}

