\section{Teorie}

Anténa je zařízení, které slouží k vysílání či přijímání elektromagnetického záření.
Každý vodič, jímž prochází střídavý elektrický proud, je vlastně anténa. Geometrickým
tvarováním vodiče a přidáváním jistých materiálů (ferity) se upravuje účinnost antény, dle
vhodných parametrů.

Každá anténa má své základní parametry, mezi hlavní patří zisk, rezonanční frekvence,
šířka přijímaného pásma a vyzařovací úhel. Podle použití antény se tyto parametry mění.
Avšak hlavním parametrem, který nejvíce ovlivňuje vlastnosti, je fyzický tvar antény.

Proto dělíme antény na dipólové, monopólové, složené (Yagi-Uda anténa), smyčkové, kónické
a clonové. Každá kategorie představuje jiné spektrum použití. Dipólové
se používají pro vysílání TV signálu. Monopólové se používají při
příjmu rádiových stanic, vlnové délky řádu desítek centietrů. Složené jsou vysoce účinné
pro širokopásmový příjem, TV i rádia. Smyčkovým anténám se budeme ještě věnovat. Kónické
jsou vysoce účiné při vysokých frekvencích. Clonové jsou ideální pro dlouhý dosah.




