\section{Teorie}

Anténa je zařízení, které slouží k vysílání či přijímání elektromagnetického záření.
Každý vodič, jímž prochází střídavý elektrický proud, je vlastně anténa. Geometrickým
tvarováním vodiče a přidáváním jistých materiálů (ferity) se upravuje účinnost antény, dle
vhodných parametrů.

Každá anténa má své základní parametry. Mezi hlavní patří zisk, rezonanční frekvence,
šířka přijímaného pásma a vyzařovací úhel. Podle použití antény se tyto parametry mění.
Avšak hlavním parametrem, který nejvíce ovlivňuje vlastnosti, je fyzický tvar antény.

Proto dělíme antény na dipólové, monopólové, složené (Yagi-Uda anténa), smyčkové, kónické
a clonové. Každá kategorie představuje jiné spektrum použití. Dipólové
se používají pro vysílání TV signálu. Monopólové se používají při
příjmu rádiových stanic, vlnové délky řádu desítek centimetrů. Složené jsou vysoce účinné
pro širokopásmový příjem, TV i rádia. Smyčkovým anténám se budeme ještě věnovat. Kónické
jsou vysoce účiné při vysokých frekvencích. Clonové jsou ideální pro dlouhý dosah.

\subsection{Smyčková anténa}

Smyčkové antény jsou velice jednoduché a všestranné. Jejich tvar je rozmanitý, například
zaujímají tvar čtvercovitý, trojúhelníkový, kruhový a elipsovitý. Kvůli jednoduchému
návrhu a analýze jsou nejvíce rozšířené mezi radioamatéry.

Klasifikují se dvou hlavních skupin, elektricky malé a velké. Elektricky malé antény jsou
ty, jejichž celková délka (počet závitů znásobený obvodem antény) je menší než desetina
vlnové délky. Velké antény svojí délkou odpovídají vlnové délce rezonanční
frekvence. Nejvíce se smyčkové antény používají v pásmech dlouhých vln.

Elektricky malé smyčkové antény mají obvykle menší radiační odpor než ztrátový odpor, tedy
jsou špatné zářiče, používají se výhradně na příjem signálu. Nejvíce se používají při
radiovému příjmu.

Radiační odpor lze zvýšit, a tím pádem zvýšit zisk antény. Nebo prodloužit elektrickou
délkou antény či
použít jádro z materiálu o vysoké permeabilitě, nejčastěji ferit. Anténa, která je tvořená
cívkou navinutou na feritovém jádře, se nazývá feritová smyčková anténa \cite{book04}.

\newpage

\section{Konstrukce}

Pro tento projekt byla zvolena feritová smyčková anténa. Hlavní důvody, proč byl
zvolen tento typ antény je kompaktnost, velikost řádu centimetrů, a zisk při dané
rezonanční frekvenci. Praktická konstrukce je též jednoduchá. Jedná se vlastně o cívku,
která je navinutá na feritové tyčince. S tím úzce souvicí pořizovací cena antény, která je
nízká. Nevýhodou bylo shánění feritové tyčinky.

Proto byla nejdříve pořízena tyčinka a podle ní byla vypočítána cívka. Pro výrobu cívky byl
použit enamelový drát o průměru 0.2mm. Na polohu cívky, počet závitů a další parametry
byl navržen program, o něm později. Cívka se nesmí na tyčce pohybovat, proto je zalitá
epoxydem.

Správná funkčnost antény byla zajištěna pomocí následující sestavy. Na generátor funkcí
byl připojen volný vodič. Vytvořili jsme
jednoduchou anténu. Anténa byla připojena na předzesilovač a poté na osciloskop. Měněním
frekvence na generátoru a sledováním úrovně napětí na výstupu zesilovače. Největší
úrovně jsme dosáhli na frekveci 77.564 kHz, což je v pořádku, neboť v
demodulačním obvodu je použit vysoce účinný krystalový filtr \cite{book01}

\subsection{Program pro výpočet antény}

Program na výpočet antény se skládá z dvou hlavních částí, kalkulace modelu antény a poté
zpracování dat do grafů.

Základem je výpočet radiačního - $R_r$, ohmického - $R_L$ a ztrátového odporu - $R_f$
\cite{book02}.

\begin{equation}
    \begin{gathered}
        R_r = 20\pi^2\left(\frac{o}{\lambda}\right)^4 \mu_{eff}^2 N^2 \\
        R_L = \frac{l}{\sigma\pi d \delta} \\
        R_f = 2 \pi f \mu_{eff} \frac{\mu_i}{\mu} \mu_0 N^2 \frac{S}{L_r}
    \end{gathered}
\end{equation}

Kde $o$ je účinný obvod antény, $N$ je počet závitů cívky, $d$ je průměr vodiče, $f$ je
rezonanční frekvence, $S$ je účinný povrch antény a $L_r$ je délka antény.

Pro výpočet pozice cívky na feritové tyčce byly použity Nagaokovy rovnice, které nejblíže
odpovídají reálnému modelu antény.

Dále program spočítá rezonanční kondenzátor k dané anténě, jakost antény. Zde je nutno
podotknout, že jakost úzce souvisí s ziskem a šířkou přijímaného pásma. Čím užší
přijímané pásmo, tím vyšší zisk, avšak anténa je více náchylná na změny tlaku, teplot atd.
Právě proto generuje program grafy, které vystihují závislosti a díky nímž je snažší
nalézt optimální parametry pro danou anténu.

Na konec program ještě vygeneruje tabulku, v níž jsou zapsány jednotlivé konfigurace
antén \cite{book03}


